\documentclass{article}
\usepackage[utf8]{inputenc}
\usepackage{graphicx}
\usepackage{amsmath}
\usepackage{listings}
\usepackage{algorithm2e}
\usepackage{fancyhdr}

\title{Triangle}
\author{Vincent Bernier}
\date{\today}

\pagestyle{fancy}
\renewcommand\headrulewidth{1pt}
\fancyhead[L]{Les triangles}
\renewcommand\footrulewidth{1pt}
\fancyfoot[R]{\today}



\begin{document}

\maketitle
\newpage
\tableofcontents
\listoffigures
\listoftables
\newpage



\section{Triangle}
    \subsection{Description}
    Le triangle est une figure plane composé de 3 sommets et 3 segments\footnote{Aussi appeler côté}.
    \begin{figure}[h]
        \centering
        \includegraphics[scale=0.5]{triangle.png}
        \caption{Exemple de triangle}
        \label{Triangle}
    \end{figure}
    \subsection{Type de triangle}
    Il existe 4 different type de triangle :
    \begin{enumerate}
        \item Triangle Rectangle
        \item Triangle équilatéral
        \item Triangle isocèle
        \item Triangle quelconque
    \end{enumerate}
    \begin{table}[h]
    \centering
    \begin{tabular}{c|c}
    Type & Particularité \\ \hline
        triangle rectangle & un angle droit \\
        triangle équilatéral & trois cotés de même longueur \\
        triangle isocèle & deux cotés de même longueur \\
        triangle quelconque & trois cotés de longueurs différentes \\
    \end{tabular}        
    \caption{Les types de triangles}
    \label{tab:my_label}
\end{table}


\newpage
\section{Triangle en math}
    \subsection{Formule}
    \begin{center}
        Pour calculer l'aire : 
        \begin{equation*}
        A = \frac{base * hauteur}{2} 
        \end{equation*}
    \end{center}
    \begin{center}
        Pour calculer Périmètre : 
        \begin{equation*}
            P = c1 + c2 + c3
        \end{equation*}
        
        \begin{equation*}
        P \leq \frac{p^2 \sqrt{3}}{36}
        \end{equation*}
    \end{center}
    \subsection{Algorithme}
    \begin{lstlisting}[language=c]
        int main() {
            int a,b,c,p;
            printf("Longeur du premier c : ");
            scanf("%d", &a);
            printf("Longeur du second c : ");
            scanf("%d", &b);
            printf("Longeur du dernier c : ");
            scanf("%d", &c);
            p = a + b + c;
            printf("p = %d", p);
            }
    \end{lstlisting}
    \newpage
    \begin{thebibliography}{1}
        \bibitem{texbook}
        Jean-Denis Eiden, Géométrie analytique classique (2009)
    \end{thebibliography}
\end{document}
